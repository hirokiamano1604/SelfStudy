\documentclass[a4paper, 11pt]{ltjsarticle}
% --- 基本設定 ---
% \usepackage{luatexja-fontspec} % lualatexで日本語フォント指定
% Macのヒラギノフォントを指定 (存在しない場合は他のフォントに変更してください)
% \setmainjfont[BoldFont={Hiragino Sans W6}, ItalicFont={Hiragino Mincho ProN W3}]{Hiragino Sans W3}
% \setsansjfont[BoldFont={Hiragino Sans W6}]{Hiragino Sans W3}
% \setmonojfont{Osaka-等幅} % Osakaフォントがない場合は他の等幅フォントを指定

% \usepackage[haranoaji]{luatexja-preset} % Harano Aji Fonts (Noto Fonts 由来) を使う場合
\usepackage[hiragino-pro,deluxe]{luatexja-preset} % ヒラギノフォントを使う場合 (より詳細な設定)

% \usepackage[T1]{fontenc} % LuaLaTeX + fontspecでは通常不要
\usepackage{amsmath, amssymb, mathtools, amsthm} % 数式関連
\usepackage{graphicx} % 画像挿入
\usepackage{hyperref} % ハイパーリンク
\usepackage{listings} % ソースコード表示
\usepackage{enumitem} % リスト環境のカスタマイズ
\usepackage{graphicx}
\usepackage{mathrsfs}
\usepackage{romanbar}

% --- Beamerテーマと配色 (記事形式では不要なため削除・コメントアウト) ---
% \usetheme{Madrid}
% \usecolortheme{default}

% カスタムカラー (tcolorboxなどで使用するため残す)
% \definecolor{myblue}{RGB}{0,110,180}
% \definecolor{mygreen}{RGB}{0,150,100}
% \definecolor{myorange}{RGB}{230,120,20}
% \definecolor{mygray}{RGB}{240,240,240}

% Beamerの配色を一部上書き (記事形式では不要なため削除・コメントアウト)
% \setbeamercolor{palette primary}{bg=myblue,fg=white}
% ... (他のsetbeamercolorも同様に削除) ...
% \setbeamercolor{block body}{bg=mygray}

% --- tcolorboxの設定 (スタイリッシュな囲み) ---
\usepackage[most]{tcolorbox}

%================================================================
% 定理・定義・命題などのカスタムボックス環境
%================================================================

% --- 必要なパッケージ(プリアンブルにあることを確認)---
% \usepackage{amsthm}
% \usepackage{amssymb}
% \usepackage[most]{tcolorbox}
% \tcbuselibrary{theorems} % \newtcbtheorem を使うために必要


% --- カスタムカラー定義 ---
\definecolor{myblue}{RGB}{0,110,180}
\definecolor{mygreen}{RGB}{0,150,100}
\definecolor{myorange}{RGB}{230,120,20}
\definecolor{mypurple}{RGB}{120, 80, 190}
\definecolor{mycautionred}{RGB}{192,0,0}
\definecolor{mycautionyellow}{RGB}{255,242,204}


% --- 自動採番される環境の定義 ---

% まず「定理」を基準として定義。番号はセクションごとにリセット
\newtcbtheorem[number within=subsection]{mytheorem}{定理}{
  breakable,
  colback=mygreen!5!white,
  colframe=mygreen!75!black,
  colbacktitle=mygreen!75!black,
  coltitle=white,
  fonttitle=\bfseries
}{thm}

% 定義・命題・補題は、「定理」と同じ番号カウンターを共有する
\newtcbtheorem[use counter from=mytheorem]{mydefinition}{定義}{
  breakable,
  colback=myblue!5!white,
  colframe=myblue!75!black,
  colbacktitle=myblue!75!black,
  coltitle=white,
  fonttitle=\bfseries
}{def}

\newtcbtheorem[use counter from=mytheorem]{myproposition}{命題}{
  breakable,
  colback=mypurple!5!white,
  colframe=mypurple!75!black,
  colbacktitle=mypurple!75!black,
  coltitle=white,
  fonttitle=\bfseries
}{prop}

\newtcbtheorem[use counter from=mytheorem]{mylemma}{補題}{
  breakable,
  width=0.9\textwidth,  % 幅を90%に
  center,               % 中央揃え
  colback=mygreen!5!white,
  colframe=mygreen!75!black,
  colbacktitle=mygreen!75!black,
  coltitle=white,
  fonttitle=\bfseries
}{lem}


% --- 手動でタイトルを指定する、または番号なしの環境 ---

% 例題用のスタイル(番号やタイトルは手動で指定)
\newtcolorbox{myexample}[2][]{
  breakable,
  colback=myorange!5!white,
  colframe=myorange!75!black,
  colbacktitle=myorange!75!black,
  coltitle=white,
  fonttitle=\bfseries,
  title={例題: #2},
  #1
}

% 注意書き用のスタイル(タイトルは「注意」で固定、番号なし)
\newtcolorbox{mycaution}[1][]{
  breakable,
  colback=mycautionyellow,
  colframe=mycautionred,
  colbacktitle=mycautionred,
  coltitle=white,
  fonttitle=\bfseries,
  title={注意},
  #1
}


\newcommand{\bb}[1]{\mathbb{#1}}
\newcommand{\mcal}[1]{\mathcal{#1}}
\newcommand{\st}{\quad \text{s.t.} \quad}
\newcommand{\prsp}{(\Omega, \mcal{F}, \bb{P})}
\DeclareMathOperator*{\argmin}{arg\,min}
\DeclareMathOperator*{\argmax}{arg\,max}

% --- 段落インデント設定 ---
\setlength{\parindent}{0pt} % デフォルトはインデントなし

\newlength{\manualparindent}
\setlength{\manualparindent}{1\zw} % ← ここが重要(1zw ではなく 1\zw)


\newcommand{\pindent}{\hspace*{\manualparindent}} % 手動で入れるインデント

\title{Modern Portofolio Theory}
\author{天野 広大}
\date{2026/1/1}

\begin{document}
\maketitle

\section{Markowitz Optimization and the Efficient Frontier}
Think of the plane whose x-axis is a variance and y-axis is a expected return.

A set of possible returns of portfolios can be denoted by a curve which pass through the original two portfolios.

By adding a neww asset to the options, it dramatically expands the range of the possible returns and variances.

The optimized portfolio should locate on the upper boundary of the region, and the upper edge is called as \textbf{Efficient Portfolio}
\section{Applying quadprog to draw the efficient Frontier}
Since we use the same plane with the first section, we have to get an expression of an expected return and a variance of the portfolio.
\[R_p = \sum_{i=1}^{k} w_i R_i\]
\[\sigma_p^2 = \sum_{i=1}^{k} \sum_{j=1}^{k} w_i w_j \sigma_i \sigma_j \rho_{ij} = \sum_{i=1}^{k} \sum_{j=1}^{k} w_i w_j \sigma_{ij}\]
where $R_i$ is an expected return for the asset $i$ and $\sigma_i$ is a variance for the asset $i$.

If there are more than 2 assets, the return of the portfolio will be 
\[R_p = w^{T} R\]
where $ w^{T}$ is a weight vector and $R$ is the asset returns.

To find the efficient frontier, we need to solve these equation
\[\min \frac{1}{2} w^{T} \Sigma w\]
\[s.t. w^{T}R = r_0, w^T = 1, w \geq 0\]

\section{Fund Separation Theorem and the Capital Market Line}
Capital Market Line

The efficient frontier changes it's shape dramatically when a risk free asset is introduced.

The \textbf{Capital Market Line} is a tangency line of the efficient frontier which pass through the risk-free rate.
This line has a mazimized sharp ratio.

Since the sharp ratio is 
\[SR_p = \frac{\mu_p - r_f}{\sigma_p} = \frac{\sum_{i=1}^{N}w_i \mu_i - r_f}{\sqrt{\sum_{i,j = 1}^{N} w_i w_j \sigma_{ij}}}\]
the weights of a tangency portfolio is given by
\[\argmax_ {w} SR_p\]
\section{Lack of robustness of Markowitz analysis}
Estimation error is a main challenge of portfolio optimization.

Some uncertaincy always exist in a parameter estimation.

In order to tackle with the estimation error, people think of the global minimum variance portfolio.
This GMV is the portfolio with the least variance in the eddicient fromtier, and is least sensitive to estimation errors.
\end{document}