\documentclass[a4paper, 11pt]{ltjsarticle}
% --- 基本設定 ---
% \usepackage{luatexja-fontspec} % lualatexで日本語フォント指定
% Macのヒラギノフォントを指定 (存在しない場合は他のフォントに変更してください)
% \setmainjfont[BoldFont={Hiragino Sans W6}, ItalicFont={Hiragino Mincho ProN W3}]{Hiragino Sans W3}
% \setsansjfont[BoldFont={Hiragino Sans W6}]{Hiragino Sans W3}
% \setmonojfont{Osaka-等幅} % Osakaフォントがない場合は他の等幅フォントを指定

% \usepackage[haranoaji]{luatexja-preset} % Harano Aji Fonts (Noto Fonts 由来) を使う場合
\usepackage[hiragino-pro,deluxe]{luatexja-preset} % ヒラギノフォントを使う場合 (より詳細な設定)

% \usepackage[T1]{fontenc} % LuaLaTeX + fontspecでは通常不要
\usepackage{amsmath, amssymb, mathtools, amsthm} % 数式関連
\usepackage{graphicx} % 画像挿入
\usepackage{hyperref} % ハイパーリンク
\usepackage{listings} % ソースコード表示
\usepackage{enumitem} % リスト環境のカスタマイズ
\usepackage{graphicx}
\usepackage{mathrsfs}
\usepackage{romanbar}

% --- Beamerテーマと配色 (記事形式では不要なため削除・コメントアウト) ---
% \usetheme{Madrid}
% \usecolortheme{default}

% カスタムカラー (tcolorboxなどで使用するため残す)
% \definecolor{myblue}{RGB}{0,110,180}
% \definecolor{mygreen}{RGB}{0,150,100}
% \definecolor{myorange}{RGB}{230,120,20}
% \definecolor{mygray}{RGB}{240,240,240}

% Beamerの配色を一部上書き (記事形式では不要なため削除・コメントアウト)
% \setbeamercolor{palette primary}{bg=myblue,fg=white}
% ... (他のsetbeamercolorも同様に削除) ...
% \setbeamercolor{block body}{bg=mygray}

% --- tcolorboxの設定 (スタイリッシュな囲み) ---
\usepackage[most]{tcolorbox}

%================================================================
% 定理・定義・命題などのカスタムボックス環境
%================================================================

% --- 必要なパッケージ(プリアンブルにあることを確認)---
% \usepackage{amsthm}
% \usepackage{amssymb}
% \usepackage[most]{tcolorbox}
% \tcbuselibrary{theorems} % \newtcbtheorem を使うために必要


% --- カスタムカラー定義 ---
\definecolor{myblue}{RGB}{0,110,180}
\definecolor{mygreen}{RGB}{0,150,100}
\definecolor{myorange}{RGB}{230,120,20}
\definecolor{mypurple}{RGB}{120, 80, 190}
\definecolor{mycautionred}{RGB}{192,0,0}
\definecolor{mycautionyellow}{RGB}{255,242,204}


% --- 自動採番される環境の定義 ---

% まず「定理」を基準として定義。番号はセクションごとにリセット
\newtcbtheorem[number within=subsection]{mytheorem}{定理}{
  breakable,
  colback=mygreen!5!white,
  colframe=mygreen!75!black,
  colbacktitle=mygreen!75!black,
  coltitle=white,
  fonttitle=\bfseries
}{thm}

% 定義・命題・補題は、「定理」と同じ番号カウンターを共有する
\newtcbtheorem[use counter from=mytheorem]{mydefinition}{定義}{
  breakable,
  colback=myblue!5!white,
  colframe=myblue!75!black,
  colbacktitle=myblue!75!black,
  coltitle=white,
  fonttitle=\bfseries
}{def}

\newtcbtheorem[use counter from=mytheorem]{myproposition}{命題}{
  breakable,
  colback=mypurple!5!white,
  colframe=mypurple!75!black,
  colbacktitle=mypurple!75!black,
  coltitle=white,
  fonttitle=\bfseries
}{prop}

\newtcbtheorem[use counter from=mytheorem]{mylemma}{補題}{
  breakable,
  width=0.9\textwidth,  % 幅を90%に
  center,               % 中央揃え
  colback=mygreen!5!white,
  colframe=mygreen!75!black,
  colbacktitle=mygreen!75!black,
  coltitle=white,
  fonttitle=\bfseries
}{lem}


% --- 手動でタイトルを指定する、または番号なしの環境 ---

% 例題用のスタイル(番号やタイトルは手動で指定)
\newtcolorbox{myexample}[2][]{
  breakable,
  colback=myorange!5!white,
  colframe=myorange!75!black,
  colbacktitle=myorange!75!black,
  coltitle=white,
  fonttitle=\bfseries,
  title={例題: #2},
  #1
}

% 注意書き用のスタイル(タイトルは「注意」で固定、番号なし)
\newtcolorbox{mycaution}[1][]{
  breakable,
  colback=mycautionyellow,
  colframe=mycautionred,
  colbacktitle=mycautionred,
  coltitle=white,
  fonttitle=\bfseries,
  title={注意},
  #1
}


\newcommand{\bb}[1]{\mathbb{#1}}
\newcommand{\mcal}[1]{\mathcal{#1}}
\newcommand{\st}{\quad \text{s.t.} \quad}
\newcommand{\prsp}{(\Omega, \mcal{F}, \bb{P})}
\DeclareMathOperator*{\argmin}{arg\,min}
\DeclareMathOperator*{\argmax}{arg\,max}

% --- 段落インデント設定 ---
\setlength{\parindent}{0pt} % デフォルトはインデントなし

\newlength{\manualparindent}
\setlength{\manualparindent}{1\zw} % ← ここが重要(1zw ではなく 1\zw)


\newcommand{\pindent}{\hspace*{\manualparindent}} % 手動で入れるインデント

\title{Beyond of diversification/ Module 3}
\date{2026/1/1}

\begin{document}
\maketitle

\section{Limits of diversification}
The benefit of the diversification is to eliminate the idiosyncratic risks.
However, there is a limit of diversification.

Under the situation like the financial crisis, every asset in the market goes down and the effect of the diversification is likely to decline.
Improving the portfolio doesn't work well.

The hedging is the only effective against the downside risk, which means that they cannnot treat the upside risk.

However, in terms of the insurance, we need to consider the upside risk as well as the downside risk.
Insurance is a kind of "Dynamic hedging", which means that we need to adjust the allocation of the assets accoding to the market situation.

\section{An introduction to CPPI}
The CPPI procedure allows for the construction of convex payoffs.

This procedure dynamically allocates total assets to a risky asset and a safe asset.

Let Floor(F) be the protection floor. This floor is the minimun of value that you can accept the loss.

The Cushion(C) is given by
\[\text{CPPI} - F\]
and the potion of investment for the risky asset is given by
\[\text{Multiplier(M)} \times C\]

For example, $when M = 3$ and the floor is $80\%$, the initial investment in the risky asset is 
\[3 \times (\$100 - \$80) = \$60\]

However, if there is a gap between the trdable dates and the huge decline in the price happens within the period, it is impossible to adjust the portfolio.
This is called as \textbf{Gap risk}.
In other words, the loss of the risky asset relative to the safe asset exceeds $\frac{1}{M}$, 

AT the same time, let's introduce the maximum drawdown constraint.

The max drawdown constraint is 
\[V_t > \alpha M_t\]
where $V_t$ is the value of the portfolio at time $t$, 
$M_t$ is the maximum value of the portfolio between time 0 and $t$
and $\alpha$ is the maximum acceptable drawdown.

The maximum drawdown flow is a continuously increasing function.

The next extension is a perfomance CAP.

Let $F_t$ be the Floor, $T_t$  be the threshold, $C_t$ be the CAP and $A_t$ be the value of the asset at the time $t$.

Here, we should have a strategy below
\[E_t = 
\begin{cases}
  &m(A_t - F_t) \quad F_t \leq A_t \leq T_t\\
  &m(C_t - A_t) \quad T_t \leq A_t \leq C_t
\end{cases}\]
where $E_t$ is the equity.
You just need to adjust your investment depending on the $A_t$.

The threshold $T_t$ is given by
\[T_t = \frac{F_t + C_t}{2}\]
which is called as \textbf{Sommoth-pasting} condition.


\section{Simulating asset returns with random walks}
The return can be modelled as 
\[\begin{cases}
  \frac{d S_t}{S_t} = (r + \sigma \lambda) dt + \sigma dW_t\\
  \frac{dB_t}{B_t} = rdt \iff B_t = B_0 e^{rt}
\end{cases}\]
where $S_t$ is the price of the stock, $B_t$ is the value of cash you hold, 
$r$ is the risk-free rate, $\sigma$ is the volatility and $\lambda$ is the sharpe ratio.

And the mean $\mu$ is given by
\[\mu = r + \sigma \lambda\]  
and the $W_t$ is the \textbf{Brownian Motion}.

This Brownian Motion is a random walk in continuous time. This one is introduced by Louis Bachelier.

In other words, the return process $S_t$ satisfies
\[\frac{S_{t+dt}-S_t}{S_t} = (r+\sigma \lambda)dt + \sigma \sqrt{dt} \xi_t\]
where $\xi_t \sim \mcal{N}(0,1)$

\section{Monte Carlo Simulation}
In general, every coefficient will change as time passes.By adding time r to each coefficient,
\[\frac{dS_t}{S_t} = \left(r_t + \sqrt{V_t} \lambda_t^S\right)dt + \sqrt{V_t} dW_t^S\]

For example, model can be like this:
\[dr_t = a(b-r_t)d_t + \sigma_r dW_t^r\]
\[dV_t = \alpha (\bar{V_t}-V_t)dt + \sigma_V \sqrt{V_t} dW_t^V\]


\end{document}